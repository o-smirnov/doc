\oddsidemargin=-5mm
\evensidemargin=-5mm
\topmargin=-5mm
\textwidth=170mm
\textheight=240mm

\pagestyle{myheadings}

% \qq{text} used to quote code (in fixed font)
\newcommand{\qq}[1]{{\tt #1}}

% \url{text} inserts a URL
\newcommand{\url}[1]{{\tt #1}}

% commands for some common Meq-terms
\newcommand{\Request}{{\tt Request}}
\newcommand{\RequestId}{{\tt RequestId}}
\newcommand{\Result}{{\tt Result}}
\newcommand{\VellSet}{{\tt VellSet}}
\newcommand{\Cells}{{\tt Cells}}
\newcommand{\Vells}{{\tt Vells}}
\newcommand{\Domain}{{\tt Domain}}
\newcommand{\Node}{{\tt Node}}
\newcommand{\Parm}{{\tt Parm}}
\newcommand{\Polc}{{\tt Polc}}
\newcommand{\RES}[1]{{\tt RES\_#1}}

% math symbols: real numbers, complex numbers, P for parameter space
\newcommand{\RR}{\mathcal{R}}
\newcommand{\CC}{\mathcal{C}}
\newcommand{\PP}{\mathcal{P}}

% define some standard colors
\definecolor{tableheading}{rgb}{0.8,0.85,1}
\definecolor{tabletext}{gray}{0}
\definecolor{tablesubheading}{gray}{.9}
%
\definecolor{highlight}{rgb}{1,1,0}

% this command creates a paragraph with a highlit background, separated
% by extra space from surrounding paragraphs.
\newcommand{\highlightbox}[1]{\smallskip\noindent\colorbox{highlight}{\parbox{\textwidth}
{\noindent #1}}\smallskip}%
% \addvspace{\smallskipamount}}

% \tablesubheading{n}{text} inserts a sub-heading into a table.
% n argument is the number of columns that the sub-heading will span

  \newlength{\taboverhang}
  \setlength{\taboverhang}{\tabcolsep}
  \addtolength{\taboverhang}{-1pt}

\newcommand{\tablesubheading}[2]%
{\multicolumn{#1}{|>{\columncolor{tablesubheading}}l|}{\strut\em~~~~~#2}}

% \standardtableheading produces a standard heading for tables
% field name | type | description
\newcommand{\standardtableheading}[1]%
{\rowcolor{tableheading}#1\\}

% \recordtableheading produces a standard heading for record tables:
% field name | type | description
\newcommand{\recordtableheading}
{\standardtableheading{\sl field name & \sl type & \sl description}}

% \statetableheading produces a standard heading for state tables:
% field name | type | default | description
\newcommand{\statetableheading}
{\standardtableheading{\sl field name & \sl type & \sl default & \sl description}}


\arrayrulecolor{tableheading}

% \recordtableentry{name}{type}{desc} produces one line
% of a record layout table
\newcommand{\recordtableentry}[3]
{\color{tabletext}\qq{#1} & \qq{#2} & #3\\}

% \statetableentry{name}{type}{default}{desc} produces one line
% of a state record layout table
\newcommand{\statetableentry}[4]
{\color{tabletext}\qq{#1} & \qq{#2} & \qq{#3} & #4\\}
