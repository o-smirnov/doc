\documentclass[12pt,twoside]{book}

\usepackage{graphicx}
\usepackage{booktabs}
\usepackage{color}
\usepackage{colortbl}

\oddsidemargin=-5mm
\evensidemargin=-5mm
\topmargin=-5mm
\textwidth=170mm
\textheight=240mm

\pagestyle{myheadings}

% \qq{text} used to quote code (in fixed font)
\newcommand{\qq}[1]{{\tt #1}}

% \url{text} inserts a URL
\newcommand{\url}[1]{{\tt #1}}

% commands for some common Meq-terms
\newcommand{\Request}{{\tt Request}}
\newcommand{\RequestId}{{\tt RequestId}}
\newcommand{\Result}{{\tt Result}}
\newcommand{\VellSet}{{\tt VellSet}}
\newcommand{\Cells}{{\tt Cells}}
\newcommand{\Vells}{{\tt Vells}}
\newcommand{\Domain}{{\tt Domain}}
\newcommand{\Node}{{\tt Node}}
\newcommand{\Parm}{{\tt Parm}}
\newcommand{\Polc}{{\tt Polc}}
\newcommand{\RES}[1]{{\tt RES\_#1}}

% math symbols: real numbers, complex numbers, P for parameter space
\newcommand{\RR}{\mathcal{R}}
\newcommand{\CC}{\mathcal{C}}
\newcommand{\PP}{\mathcal{P}}

% define some standard colors
\definecolor{tableheading}{rgb}{0.8,0.85,1}
\definecolor{tabletext}{gray}{0}
\definecolor{tablesubheading}{gray}{.9}
%
\definecolor{highlight}{rgb}{1,1,0}

% this command creates a paragraph with a highlit background, separated
% by extra space from surrounding paragraphs.
\newcommand{\highlightbox}[1]{\smallskip\noindent\colorbox{highlight}{\parbox{\textwidth}
{\noindent #1}}\smallskip}%
% \addvspace{\smallskipamount}}

% \tablesubheading{n}{text} inserts a sub-heading into a table.
% n argument is the number of columns that the sub-heading will span

  \newlength{\taboverhang}
  \setlength{\taboverhang}{\tabcolsep}
  \addtolength{\taboverhang}{-1pt}

\newcommand{\tablesubheading}[2]%
{\multicolumn{#1}{|>{\columncolor{tablesubheading}}l|}{\strut\em~~~~~#2}}

% \standardtableheading produces a standard heading for tables
% field name | type | description
\newcommand{\standardtableheading}[1]%
{\rowcolor{tableheading}#1\\}

% \recordtableheading produces a standard heading for record tables:
% field name | type | description
\newcommand{\recordtableheading}
{\standardtableheading{\sl field name & \sl type & \sl description}}

% \statetableheading produces a standard heading for state tables:
% field name | type | default | description
\newcommand{\statetableheading}
{\standardtableheading{\sl field name & \sl type & \sl default & \sl description}}


\arrayrulecolor{tableheading}

% \recordtableentry{name}{type}{desc} produces one line
% of a record layout table
\newcommand{\recordtableentry}[3]
{\color{tabletext}\qq{#1} & \qq{#2} & #3\\}

% \statetableentry{name}{type}{default}{desc} produces one line
% of a state record layout table
\newcommand{\statetableentry}[4]
{\color{tabletext}\qq{#1} & \qq{#2} & \qq{#3} & #4\\}


\markboth{\dotfill MeqTree Browser Interface}{\small\tt
LOFAR/Timba/doc/MEQ/PythonInterface.tex\em~~$\circ$~$ $Revision$ $~$\circ$~$ $Date$ $\dotfill}

\title{{\sf The MeqTree Browser Inteface\\\small or Do Pythons Eat Timbas?}}

\author{{\sf O.M. Smirnov}}

\date{\vspace{2cm}\small CVS path: \tt LOFAR/Timba/doc/PythonInterface.tex\\\rm
$ $Revision$ $\\$ $Date$ $}

\begin{document}

\sloppy

\maketitle
% \tableofcontents

\chapter*{Overview}

  A plug-in is a class that extends MeqBrowser functionality. Several kinds
  of plug-ins may be created:

  \begin{description}
  
  \item[Viewer plug-ins] (e.g. Record Browser, Result Plotter) are associated
  with data types, and are activated when a user asks to display a data object
  (e.g. from the ``Display with'' context menu.)

  \item[Node actions] (e.g. MeqParm Fiddler, Reexecutor) are associated with
  nodes (either all nodes or specific node classes), and are available via the
  node context menu.

  \item[Other plugins] (to be implemented) are not associated with a specific
  data object or node, and can be activated via the main menu.

  \end{description}

  The first chapter of this document features a primer on writing MeqBrowser
  plug-ins, with some hands-on examples. This will briefly touch on most aspects
  of the browser API; a detailed reference is provided in subsequent chapters
  (and if they're missing, it is simply because I haven't written them yet). 
  
\chapter{Writing Browser Plug-ins: A Primer}
  
  A plug-in is generally just a Python class/module which is imported into the
  browser. All plug-ins are free to create and display any kind of PyQt widget,
  and can interact with the kernel via the \qq{meqds} (Meq Data Services)
  module. Getting a plug-in to fully integrate into the browser (i.e.,  play
  nice with the gridded workspace, etc.) takes a few extra incantations,
  however. Hopefully, these will become clear by the end of this chapter.

\section{Viewer Plugins}

  Viewers are associated with data object. The lifecycle of a viewer plug-in is
  typically as follows:
  
  \begin{enumerate} 
  
  \item The user clicks on a data object (i.e. default viewer wanted), or
  selects a viewer from the ``Display with...'' context menu.
  
  \item The browser creates a \qq{Timba.Grid.DataItem} object. A \qq{DataItem} 
  represents the data object being viewed. The \qq{Timba.Grid.addDataItem()} 
  function is called to register the dataitem with the grid.
  
  \item Certain dataitems are considered {\em mutable}; this goes for, e.g.,
  node state, or any part thereof. For such items, the browser will send out a
  refresh request to the kernel. Meanwhile, the dataitem may be constructed {\em
  with no data.} Thus some viewers must be prepared to receive an ``empty''
  dataitem on startup, and to be updated with data dynamically.

  Other data objects are immutable, and are available to begin with (e.g.,
  anything from the result log or the message log). For such objects, the
  dataitem is initialized with contents immediately.

  \item Within \qq{addDataItem()}, a viewer object is constructed, and assigned
  to \qq{dataitem.viewer\_obj}. The viewer can then allocate one or more cells
  in the grid, etc. (But note that it doesn't have to -- it's perfectly possible
  to have viewers outside the grid.)

  \item The browser maintains a list of currently viewed dataitems. Whenever 
  one is refreshed (e.g., node publishes state, etc.), the corresponding viewer
  object's \qq{set\_data()} method is called to update the viewer.

  \item When the user closes a viewer (e.g. with the cell close button),
  \qq{Timba.Grid.removeDataItem()} is called. This deletes the dataitem and the
  viewer object.

  \end{enumerate}
  
\section{Anatomy Of A Simple Viewer}

  Let's examine a typical viewer plug-in, the \qq{ArrayBrowser}. This is a very simple viewer, in
  that it allocates itself a single grid cell, and does not directly interact
  with the kernel at all, relying on the grid to keep it updated. Viewers like
  this can be conveniently derived from the \qq{GriddedPlugin} class
  (\qq{GUI/browsers.py}). 
  
  The Array Browser is defined in \qq{Plugins/array\_browser.py}:

\begin{verbatim}  
from Timba.dmi import *           # basic DMI support
from Timba import utils           # various utility classes
from Timba import dmi_repr        # to-string conversion functions
from Timba.GUI.pixmaps import pixmaps  # icons
from Timba import Grid            # Grid services
from Timba.GUI.browsers import \
  GriddedPlugin,PrecisionPopupMenu  # 
from qt import *                  # Qt widgets
from qttable import *             # QTable and company
\end{verbatim}

  These statement cause various useful modules to be imported. We won't discuss
  them in any detail, since these modules are (or should be...) documented
  elsewhere... and when writing your own viewer, you can always cut-and-paste.
  
\subsection{Debug printing} 

  Further down, we see:
  
\begin{verbatim}  
_dmirepr = dmi_repr.dmi_repr();

_dbg = utils.verbosity(0,name='array_browser');
_dprint = _dbg.dprint;
_dprintf = _dbg.dprintf;
\end{verbatim}
  
  The \qq{\_dmirepr} is simly a helper object that the Array Browser uses to
  convert numbers to strings. The \qq{\_dbg} object is more intersting: this is
  a debug-print object, also known as a {\em verbosity context}. Each verbosity
  context has a name and a debug level (in this case initialized to 0 by
  default). All over the browser code, you will see statements like:

\begin{verbatim} 
    _dprint(3,"rank is",self._rank);
\end{verbatim}

  This statement will conditionally print a diagnostics line on stdout, but only
  if the current debug level of the corresponding verbosity context is no lower 
  than the first argument to \qq{\_dprint}. The debug level can be set from the
  command line via the \qq{"-d"} option, e.g.: \qq{"-darray\_browser=3"}. This
  is a handy facility for adding debug printing with various levels of detail to
  your code, and you should try to use it as much as possible (in fact, loose
  \qq{print} statements are very much frowned upon...) The \qq{\_dprintf}
  function is similar, but it takes a format string as the second argument,
  e.g.:

\begin{verbatim}  
    _dprintf(3,"rank is %d",self._rank);
\end{verbatim}

  What makes this facility even more convenient is that \qq{\_dprint}
  automatically prefixes each line by the context name, filename, line number
  and function name. The statement above actually prints something like:

\begin{verbatim}  
    array_browser(array_browser.py:43:set_array): rank is 2
\end{verbatim}

  Note finally that global symbols prefixed with \qq{"\_"} are normally not
  imported into other Python modules even when the \qq{from} ... \qq{import *}
  statement is used. This is useful to hide ``internal'' module-level
  variables.

\subsection{Definining the viewer class} 

  Next, we see:

\begin{verbatim}  
class ArrayTable(QTable):
# lots of clever but irrelevant code
\end{verbatim}

  This is a specialization of the QTable widget for efficiently displaying
  numarrays. This is the main widget of the Array Browser, but its internals are
  not terribly relevant to this discussion, so we won't describe it (although
  you're free to use it in your plugins too). The viewer class itself is defined
  further down:

\begin{verbatim}  
class ArrayBrowser(GriddedPlugin):
  _icon = pixmaps.matrix;
  viewer_name = "Array Browser";
  def is_viewable (data):
    try: return 1 <= data.rank <=2;
    except: return False;
  is_viewable = staticmethod(is_viewable);
\end{verbatim}

  This derives the viewer from \qq{GriddedPlugin}, and defines an icon for the
  class, and an official viewer name (which will appear in menus and such). Note
  that \qq{GriddedPlugin} is a very ``thin'' wrapper around its parent, the
  \qq{Grid.CellBlock} class (\qq{Grid/CellBlock.py}). In fact, all of the
  methods discussed below are actually defined by \qq{CellBlock}. Keep this in
  mind, as we'll be seeing \qq{CellBlock} again elsewhere in this document.
  
\subsection{Can we view this?} 

  The \qq{is\_viewable()} method is called by the browser to determine if the
  viewer can display a specific data object or not. Viewers are associated with
  particular data types to begin with (more on this below), so if your viewer is
  so wonderfully flexible that it can handle any object of the associated
  type whatsoever, it doesn't need to define \qq{is\_viewable()} at all. Note
  also that \qq{is\_viewable()} has to be a static method, as it is called
  before any viewer object is actually instantiated.

  In our case, \qq{ArrayBrowser} is registered as a viewer for arrays. However,
  not all arrays are viewable with this class (as of now, at least), but only
  one- and two-dimensional ones. The \qq{is\_viewable()} method checks for
  this\footnote{the exception handler is there as a fallback, in case \qq{data}
  doesn't have a rank attribute at all, which is not supposed to happen, but
  that's no reason to fall over disgracefully...}; the browser will
  automatically omit the viewer from consideration if the data object is
  reported as not viewable. 

\subsection{Initialization} 

  All viewers must define a certain form of constructor. This is documented in
  \qq{Grid/Services.py}.

\begin{verbatim}  
  def __init__(self,gw,dataitem,cellspec={},**opts):
    GriddedPlugin.__init__(self,gw,dataitem,cellspec=cellspec);
    self._arr = None;
    self._tbl = ArrayTable(self.wparent());
\end{verbatim}

  Here, we don't really need to worry about most of the constructor arguments,
  since they're passed up to \qq{GriddedPlugin} (and ultimately \qq{CellBlock})
  as is. Do note the \qq{self.wparent()} call though -- the \qq{CellBlock}
  constructor allocates a grid cell for us, and the \qq{self.wparent()} method
  returns the widget corresponding to that cell. This widget should be used as
  the parent  of all the browser's widgets.

\begin{verbatim}  
    self.set_widgets(self.wtop(),dataitem.caption,icon=self.icon());
    QObject.connect(self.wtop(),PYSIGNAL("fontChanged()"),self._tbl.reset_colsizes);
\end{verbatim}

  This first initializes the grid cell, by suppling it our widget, a caption
  (one will have already been created for us in \qq{dataitem}) and an
  icon.\footnote{\qq{icon()} is defined in \qq{GriddedBrowser}, and simply
  returns our static \qq{\_icon} attribute initialized above.} Note that our
  \qq{wtop()} method (defined below) returns the top-level widget of the
  viewer -- \qq{self.\_tbl} in this case. Defining \qq{wtop()} to return the
  top-level widget of a class is a convention that is followed throughout the
  browser, I suggest you stick to it in your own plug-ins.

  The \qq{"fontChanged()"} signal is emitted on behalf of our top-level widget
  when the user changes the font size from the cell menu. The browser will
  change the Qt \qq{font} property for us automatically before this signal is
  emitted. If additional operations are required to adapt to the new font (e.g.,
  resizing the table columns), this signal provides the necessary hook.

\begin{verbatim}  
    if dataitem.data is not None:
      self.set_data(dataitem);
\end{verbatim}

  The \qq{dataitem} may or may not be already initialized with actual data
  content (the \qq{data} attribute). If it is, we call our \qq{set\_data()}
  method to fill the table widget. If it isn't (\qq{data} is \qq{None}), then
  the browser will call our \qq{set\_data()} method later, when the data 
  becomes available.
        
\begin{verbatim}  
    context_menu = self.cell_menu();
    if context_menu is not None:
      context_menu.insertSeparator();
      menu = PrecisionPopupMenu(context_menu,prec=self._tbl.get_prec());
      context_menu.insertItem(pixmaps.precplus.iconset(),'Number format',menu);
      QWidget.connect(menu,PYSIGNAL("setPrecision()"),\
                      self._tbl.set_precision);
\end{verbatim}

  The \qq{cell\_menu()} method returns the menu object associated with our grid
  cell. The cell menu has a few entries standard to all cells, but viewers are
  free to extend it with custom functions. Here, we add an item to change the
  number representation and precision; this is conveniently provided by the
  \qq{PrecisionPopupMenu} class from \qq{GUI/browsers.py}.
  
  And that's pretty much it for the constructor. There's very little code left
  in fact...

\subsection{Reporting errors} 

  Any errors during viewer initialization should be thrown as exceptions. The
  browser will catch these, report them to the user,  and go on without
  allocating any cells (and if it doesn't, then this is a bug that should be
  reported!) Thus, Array Browser does not bother itself with any error checks,
  trusting Python to raise exceptions if anything goes wrong.  If your viewer
  runs into any trouble, you should simply raise an exception.

\subsection{Updating the data} 

  All viewers must define a \qq{set\_data()} method. This is called by the
  browser whenever the data object associated with the viewer is updated.

\begin{verbatim}  
  def set_data (self,dataitem,**opts):
    self._tbl.set_array(dataitem.data);
    self.enable();
    self.flash_refresh();
\end{verbatim}
  
  The \qq{dataitem} argument is a complete dataitem, but you can always  assume
  that it's the same dataitem that you were initialized with, except that the
  \qq{dataitem.data} field is different. (The \qq{**opts} argument is provided
  for future expansion). Here, we pass the data itself to our table widget. The
  \qq{enable()} method ensures that our cell is enabled (it's normally disabled until
  data becomes available; there's no harm in calling this method every time),
  and the \qq{flash\_refresh()} call causes the ``Refresh'' toolbutton of the
  cell to flash briefly, as a visual cue to the user that viewer contents have
  been updated.

\subsection{Registering the viewer} 

  This concludes the Array Browser class definition. The only remaining step is
  to register the class with the grid services:

\begin{verbatim}  
# Give ArrayBrowser slightly higher priority for arrays
Grid.Services.registerViewer(array_class,ArrayBrowser,priority=15);
\end{verbatim}

  This associates our viewer with a specific data type (\qq{array\_class} in
  this case, which is defined in \qq{Timba.dmi} as equivalent to
  \qq{numarray.array}). The \qq{priority} argument determines the order in which
  viewers are sorted in the menus, lower numbers meaning higher priority. The
  default viewer, i.e. the one that is activated with left-click, is the one
  with the lowest number (for arrays, this is probably the \qq{ArrayPlotter}).
  Note that there's nothing wrong with registering the same viewer for several
  data types, as long as it can handle them all. 

  One final wrinkle on this is exemplified by \qq{Plugins/result\_plotter.py}:
  
\begin{verbatim}  
Grid.Services.registerViewer(dmi_type('MeqResult',record),ResultPlotter,priority=10)
\end{verbatim}

  Here, the \qq{Timba.dmi.dmi\_type()} function is used  to form a {\em derived
  DMI type}. A lot of important kernel objects are represented in Python by
  classes which are derived from base types such as arrays (from \qq{numarray})
  and \qq{record}s (from \qq{Timba.dmi}) but are otherwise identical to the base
  types as far as Python is concerned. These include MeqDomain, MeqCells,
  MeqRequest, MeqResult, MeqVellSet, MeqVells, etc. The conversion layer
  formally assigns distinct types to these objects, so that they may be
  identified as such where necessary. 

  The \qq{dmi\_type()} function returns a type object corresponding to a derived
  type. The call above then registers \qq{ResultPlotter} as a viewer for records
  of the \qq{MeqResult} variety.

  In a similar vein, the state records of individual nodes are artifically
  assigned a Python class according to their node class. The \qq{Timba.meqds}
  module provides these classes:
  
\begin{verbatim}  
  meqds.NodeClass('MeqParm')
\end{verbatim}

  for example, will return a type object representing the MeqParm node state.
  
\section{Node Actions}

  {\em Node actions} are commands associated with nodes. Node actions are placed
  into the Tree Browser toolbar and/or the node context menu and/or the main
  menu (not yet implemented). A node action can do pretty much anything it
  likes, including displaying new windows or dialogs, creating new viewers, and
  interacting with the kernel. 

\subsection{A simple node action}

  The simplest examples of node actions are provided in \qq{GUI/treebrowser.py}.
  For example, the ``Publish'' action is defined as follows:
  
\begin{verbatim}  
class NA_NodePublish (NodeAction):
  text = "Publish";
  iconset = pixmaps.publish.iconset;
  def activate (self,node):
    cmd = record(nodeindex=node.nodeindex,get_state=True,enable=not node.is_publishing());
    mqs().meq('Node.Publish.Results',cmd,wait=False);
  def is_checked (self,node):
    return node.is_publishing();
\end{verbatim}

  This defines a new node action class, derived from the \qq{NodeAction} base
  class (also found in \qq{GUI/treebrowser.py}). An instance of the class will
  be created whenever this action is inserted into a menu or a toolbar. Some
  interesting features of this class are:
  
  \begin{itemize}
  
  \item The \qq{text} and \qq{iconset} attributes define the name and icon of
  the menu entry. An optional attribute (not used here) is called
  \qq{nodeclass}. When set to a node class (as returned by
  \qq{meqds.NodeClass()}, see above) or a string class name, it restricts the
  action to nodes of a specific class. This is actually determined by the
  \qq{applies\_to\_node()} method defined at the \qq{NodeAction} level;
  subclasses may choose to redefine this method if discriminating by a single 
  node class is not specific enough.

  \item The \qq{activate()} method is called whenever the action is selected 
  from the menu. The \qq{node} argument is a \qq{Node} object (from
  \qq{Meq/meqds.py}) that defines which node the action is being applied to.
  (\qq{Node} objects are discussed in detail later on; for the moment do note
  that they are {\bf not} the same thing as a node state records, although they
  do contain some similar attributes.) The implementation here simply issues a
  command to the kernel to enable or disable publishing for the node. 

  \item The \qq{is\_checked()} method should be defined only if the action is a
  toggle. It is called whenever a menu containing the action is about to be
  displayed. The menu item is displayed as toggled-on or -off according to what
  this method returns. The default implementation (in \qq{NodeAction}) returns
  \qq{None}, indicating that the action is not a toggle.

  \item Similarly, a method called \qq{is\_enabled()} -- not redefined here --
  is called to determine whether a menu action should be enabled. The default
  implementation returns \qq{True}.

  \end{itemize}

\subsection{More node actions}

  As an example, we will review the ``Reexecute'' plug-in from
  \qq{Plugins/node\_execute.py}. This is a kind of a hybrid plug-in, as it is
  combines a viewer and a node action. The viewer is discussed separately below.

  The \qq{node\_execute} module begins much like any other plug-in, with a bunch
  of import statements and a verbosity context. Then, it defines its own node
  action:

\begin{verbatim}  
class NA_NodeExecute (NodeAction):
  text = "Reexecute";
  iconset = pixmaps.reexecute.iconset;
  def activate (self,node):
    Grid.Services.addDataItem(makeNodeDataItem(node,viewer=Executor));
  def is_enabled (self,node):
    # available in idle mode, or when stopped at a debugger. 
    # If debug_level is set, node must be idle (otherwise, we can't trust the
    # node control status to be up-to-date)
    return ( self.tb.is_stopped or not self.tb.is_running ) and \
           ( not self.tb.debug_level or node.is_idle() );
\end{verbatim}

  When activated, this action creates a dataitem for the node, and adds an
  \qq{Executor} viewer (see below) for it to the gridded workspace. Note also
  that the action is enabled only under specific circumstances -- for example,
  you cannot re-execute a node that is stopped in the tree debugger! The
  \qq{self.tb} attribute refers to the \qq{TreeBrowser} object where this node
  action resides.

\subsection{Registering custom actions}

  A plug-in that defines its own node actions must have a mechanism for
  registering them. At the bottom of \qq{node\_execute.py} you will find the
  following code:
  
\begin{verbatim}  
def define_treebrowser_actions (tb):
  _dprint(1,'defining node-execute treebrowser actions');
  tb.add_action(NA_NodeExecute,30,where="node");
\end{verbatim}

  This adds our custom action to a tree browser (given by the \qq{tb} object).
  Whenever a tree browser object is initialized, it scans all loaded modules for
  \qq{define\_treebrowser\_actions} methods, and then calls them all with itself
  as the \qq{tb} argument. 
  
  The second argument to \qq{add\_action()} is a priority that determines 
  the placement of node actions in menus (higher numbers mean lower placement).
  The \qq{where} argument determines which menu the action is actually added to,
  currently recognized values are \qq{"node"} for the node context menu, and
  \qq{"debug"} for the Debug sub-menu. 
  
  It is also possible to add custom actions to the treebrowser toolbar. At the
  moment, the toolbar will only support standard Qt \qq{QAction}s, not
  \qq{NodeActions}. To add a custom toolbar action, you can create a
  \qq{QAction} object, and pass it to \qq{tb.add\_action()} with
  \qq{where="toolbar"}. You may also assign an optional function attribute to
  the \qq{QAction} called \qq{\_is\_enabled}, this will be called with the
  treebrowser as its argument to determine whether the action should be enabled
  or not. E.g.:
  
\begin{verbatim}  
  myaction = QAction(tb.wtop());
  myaction.setText("My Action");
  myaction.setIconSet(myicon);
  myaction.connect(myaction,SIGNAL("activate()"),my_action_callback);
  # myaction_is_enabled is a function taking a single "tb" argument
  myaction._is_enabled = myaction_is_enabled; 
  tb.add_action(myaction,30,where="toolbar");
\end{verbatim}  
  

\section{A More Complicated Viewer}

  The \qq{Executor} viewer launched by the Reexecute action illustrates a number
  of interesting issues.
  
\subsection{Two data types?}
  
  First of all, at the bottom of \qq{node\_execute.py}, we note that the viewer
  is registered for two types of data objects:
  
\begin{verbatim}  
Grid.Services.registerViewer(_request_type,Executor,priority=25);
Grid.Services.registerViewer(meqds.NodeClass(),Executor,priority=25);
\end{verbatim}  

  The first one, \qq{\_request\_type}, is simply the \qq{MeqRequest} type
  (initialized to \qq{dmi\_type("MeqRequest",record)} at the top of the file).
  The second one is the generic node state record class. It follows that the
  viewer must be able to handle both types of data objects. For MeqRequest
  objects this is quite straightforward; but given a node state record,  we must
  be able to extract a request object somehow. All nodes store their most recent
  request into the \qq{request} field. Hence, we define the \qq{is\_viewable()}
  method to check that this is actually present:

\begin{verbatim}  
  def is_viewable (data):
    return isinstance(data,_request_type) or \
      isinstance(getattr(data,'request',None),_request_type);
  is_viewable = staticmethod(is_viewable);
\end{verbatim}  

  Now, registering ourselves as a viewer for node state raises a few interesting
  implementation issues. What happens if the node publishes a state update, for
  instance? The browser will happily call our \qq{set\_data()} method, but do we
  really want to update our request object? The answer is probably no: the user
  is looking at a the request, and he doesn't expect it to change under him
  unprovoked. On the other hand, when the user has manually requested a refresh
  of the loaded request, we do want to update it. \qq{Executor} implements this
  behaviour by manipulating a flag called \qq{self.\_has\_data}, see the code for
  details. 

\subsection{Reusing a record browser}

  A request object is, at core, a record. Fortunately, we already have a class
  for displaying records that we can reuse. This is the same  \qq{HierBrowser}
  class that the standard RecordBrowser viewer uses:
    
\begin{verbatim}  
    udi_root = dataitem.udi;
    if not udi_root.endswith('/request'): # not a direct request object
      udi_root += '/request';
    self.reqView = browsers.HierBrowser(self.wtop(),"","",udi_root=udi_root);
    QObject.connect(self.reqView.wtop(),PYSIGNAL("displayDataItem()"),
                    self.wtop(),PYSIGNAL("displayDataItem()"));
    self.reqView.set_refresh_func(dataitem.refresh_func);
\end{verbatim}  

  The extra incantations are necessary to get HierBrowser to play nicely with
  the grid displays (i.e., to display ``looking glass'' icons next to
  sub-fields, and to launch new viewers from them). We could also do without
  them and simply create a HierBrowser object as is, but then display of
  sub-items would not be available.

  The record browser is then populated with content later on in
  \qq{set\_data()}:
  
\begin{verbatim}  
    self.reqView.set_content(self._request);
    self.reqView.set_open_items(self.defaultOpenItems);
\end{verbatim}  

\subsection{Talking to the node}

  Once the user presses the big ``Execute'' button at the bottom of our viewer, we
  must send a command to our node. This in itself is easy enough:

\begin{verbatim}  
  def _reexecute (self):
    if not self._request:
      return;
    _dprint(1,'accepted: ',self._request);
    cmd = record(nodeindex=self._node,request=self._request,get_state=True);
    mqs().meq('Node.Execute',cmd,wait=False);
\end{verbatim}  

  However, how do we know which node to send the command to? The answer lies in
  the following piece of code from \qq{\_\_init\_\_()}:

\begin{verbatim}  
    # determine target node, form a caption and set contents
    (name,ni) = meqds.parse_node_udi(dataitem.udi);
    caption = "Reexecute <b>%s</b>" % (name or '#'+str(ni),);
    self._node = ni;
    # setup widgets
    self.set_widgets(self.wtop(),caption,icon=self.icon());
\end{verbatim}  

  Here, we use the {\em Uniform Data Identifier} (UDI) of our dataitem to
  determine the node. UDIs are strings that uniquely identify a data object, and
  every dataitem has one. UDIs will be documented in detail later on, for the
  moment, just note that the UDI of a node state record looks something like
  \qq{"/node/solver\#23"}, while the UDI of a sub-field looks something like
  \qq{"/node/solver\#23/request"}. In both cases, the node name and nodeindex
  can be extracted from the UDI, and this is exactly what
  \qq{meqds.parse\_node\_udi()} does for us.
  
  Note also that this code creates a custom caption string for our cell, using
  Qt's Rich Text markup tags. 

\subsection{Dealing with \qq{set\_data}}
  
  Besides the trickery with \qq{\_has\_data}, \qq{set\_data()}, has some more
  code to deal with our two kinds of data objects:

\begin{verbatim}  
      req = dataitem.data;
      # data is not a request, must be node state then
      if not isinstance(req,_request_type):
        req = getattr(req,'request',None);
\end{verbatim}  
  
  If there's no such field in the node state, \qq{getattr()} will return
  \qq{None}. We then do a final check to see that we have a request object 
  to display, and if not, then we report this in one of two ways:

\begin{verbatim}  
        # if no request in node state, report this
        if not isinstance(req,_request_type): 
          if self._request is not None:
            QMessageBox.warning(self.wtop(),'Missing request',
                "No request field found in node state, keeping old request",
                QMessageBox.Ok,QMessageBox.NoButton);
          else:
            self.reqView.wlistview().setRootIsDecorated(False);
            self.reqView.clear();
            QListViewItem(self.reqView.wlistview(),'','','(no request found in
            node state)');
          return;
\end{verbatim}  

  In the first case, (\qq{if self.\_request is not None}), we're already
  displaying an older request in the viewer (possible if, e.g., the user has
  asked for a refresh, but the new node state doesn't have a request in it for
  some reason), so we display a message box. Othewise, there's no request being
  displayed to begin with, so we can display an error message directly in the
  request browser.

  These extra checks may strike you as somewhat redundant given that the
  \qq{is\_viewable()} method does something very similar. Remember though that
  \qq{is\_viewable()} is only called once, before a viewer is created. Allowing
  the user to refresh a request from the node state adds extra complications
  that \qq{is\_viewable()} cannot check for.
  
\chapter{Working With Data Structures}

  straightforward...

\chapter{The Meq Data Services Module}  

  extensive...

\chapter{GUI Services And Other Useful Stuff}

  highly useful...

\end{document}
