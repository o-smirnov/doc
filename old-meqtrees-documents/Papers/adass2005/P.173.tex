\documentclass[11pt,twoside]{article}  % Leave intact
\usepackage{adassconf}
\begin{document}   % Leave intact


\paperID{P173}

\title{Implementing Arbitrary Measurement Equations With The MeqTree Module}
\titlemark{Implementing Arbitrary MEs With MeqTrees}

\author{O.M. Smirnov \& J.E. Noordam}
\affil{ASTRON, P.O. Box 2, 7990AA Dwingeloo, The Netherlands}

\contact{Oleg Smirnov}
\email{smirnov@astron.nl}

\paindex{Smirnov, O.M.}
\aindex{Noordam, J.E}     % Remove this line if there is only one author

\authormark{Smirnov \& Noordam}

%-----------------------------------------------------------------------
%			Subject Index keywords
%-----------------------------------------------------------------------
% Enter a comma separated list of up to 6 keywords describing your
% paper.  These will NOT be printed as part of your paper; however,
% they will be used to generate the subject index for the proceedings.
% There is no standard list; however, you can consult the indices
% for past proceedings (http://adass.org/adass/proceedings/).
%
% EXAMPLE:  \keywords{visualization, astronomy: radio, parallel
%                     computing, AIPS++, Galactic Center}
%
% In this example, the author noticed that "radio astronomy" appeared
% in the ADASS VII Index as "astronomy" being the major keyword and
% "radio" as the minor keyword.  The colon is used to introduce another
% level into the index.

\keywords{astronomy: radio, calibration, selfcal, MeqTrees, interferometry,
polarimetry}

\begin{abstract}           

A Measurement Equation (ME) is a mathematical model of an instrument (e.g. a
radio telescope) and the observed object(s). It can be used to predict  the data
measured with the instrument. In general, calibration of an observation (e.g.
selfcal in radio astronomy) involves solving for parameters of an ME in some
shape or form, by comparing observed and predicted data. Most calibration
packages implicitly or explicitly implement some specific form of the ME. By
sheer necessity, this involves making some simplifying assumptions about the
instrument and the observation. As new instruments come on-line (WSRT LFFEs,
LOFAR, etc.) we find ourselves pushing the limits of these assumptions. The
MeqTree module provides a flexible system to implement MEs of arbitrary
structure and complexity, and to solve for arbitrary subsets of their
parameters. 

\end{abstract}

\section{Introduction}

Measurement Equations (MEs) play a crucial role in the calibration of
interferometric observations, since the latter always involve some sort of 
model fitting. The ME of an interferometer was derived in its closed form by
Hamaker et al. (1996). For practical calibration purposes, however, the ME needs
to be expressed in a computable form, employing a limited number of
parameters. Calibration with an ME then follows a conventional model-fitting
loop: initialize parameters from apriori guesses --- use the ME to predict
observables --- compare to data --- adjust parameters --- repeat to a
(hopefully) satisfactory fit.

All existing radiointerferometry calibration packages implicitly or explicitly
implement some simplified and {\em fixed} form of the ME. The traditional {\em
selfcal} algorithm is a good example. In its simplest formulation, it models the
sky by a single point source, and the instrument by a single complex gain/phase
term per each antenna. Despite the apparent simplicity of this model, it has
served radioastronomers in good stead for over two decades!

However, a fixed ME makes it very hard or impossible to calibrate for effects
unaccounted for by the software designer. Even older instruments can push the
envelope of existing packages; future instruments such as LOFAR and SKA make the
situation worse by magnitudes. These instruments will require sophisticated
models with many ME parameters which current packages are simply not equipped to
handle. Moreover, at the moment we can only guess which formulations
of the ME will actually allow for calibration, so we will need to continue
experimenting with different MEs as these instruments come on-line. Clearly,
what is needed is a {\em toolkit} for constructing and fitting MEs. 

\section{The MeqTree Concept}

The {\em MeqTree} (Measurement Equation Tree) module provides such a toolkit.
Any model or ME is, in the final analysis, nothing more than a mathematical
expression, and as such can be represented by a tree.

MeqTrees are constructed out of {\em MeqNodes}. Nodes receive {\em MeqRequests}
from their parents and pass them on to their children, get {\em MeqResults} in
return, perform some operation on them, and return the result of that to their
parents. A MeqRequest generally defines a domain and gridding in {\em N-}D space
(for example, time-frequency space), and a MeqResult represents a sampling of
some function over that domain, {\em with optional perturbed values.}

A typical MeqNode performs some mathematical operation on the results of its
children. MeqTrees provide a large collection of node classes for most
mathematical operations, and new node classes may be added by writing a (usually
simple) C++ class. A vital feature of MeqTrees is that tensors are represented
in an elegant and economical manner. This allows for a mathematically complete
model of polarization effects (Hamaker 2000), which have up to now been
notoriously difficult to understand and calibrate for.

Thus, we can build a tree representing any function or model or ME, and evaluate
that model simply by giving a request to the root node of the tree, and waiting
for a result to come back. {\em If you can write it down as an expression, you
can predict it with a MeqTree.}

\subsection{Solving For MEs}

Building an ME is just half the job, what about fitting it to the observed data?
MeqTrees provide some specialized node classes to facilitate this:

{\bf MeqParm} leaf nodes represent ME parameters. These can be
atomic, or can in turn be functions of, e.g., frequency and time (Mevius et
al.\  2006, this volume). Any subset of MeqParms may be designated as solvable;
solvable MeqParms will return perturbed values along with their main value.

{\bf CondEq} (Conditioning Equation) nodes compare the results of their
two children --- e.g., a predict subtree on one side, and observed data  supplied
by a {\em Spigot} leaf on the other --- and convert perturbed values into numeric
derivatives.

A {\bf Solver} node does the actual fitting. It collects residuals and
derivatives from all its CondEq children, builds a Jacobian matrix, and executes
one step of the Levenberg-Marquant minimization algorithm. This results in a set of {\em
parameter updates}, which are then sent back up to the solvable MeqParms. The
cycle is repeated until a satisfactory fit is reached, or for some maximum
number of iterations. The Solver node then returns a result.

{\bf Control nodes} such as the {\bf ReqSeq} (Request Sequencer) provide flow
control. For example, multiple Solvers may be fired in series to solve for
different sets of paremeters in turn, a subtract branch can be activated,
subtracting the fitted model from the data, a correction branch may apply
derived corrections to the residuals, etc.

{\bf Sink} nodes are found at the root of trees. Sinks write their
childrens' results out to disk (e.g., to an AIPS++ MS). Depending
on the structure of the tree, these results may represent predicted
visibilities, residuals, corrected data, etc.

In this way, MeqTrees allow one to solve for arbitrary parameters of any ME. To
elaborate on a previous statement, {\em if you can write it down as an
expression, you can solve for it using MeqTrees.} (Of course, the data needs to
provide enough constraints to solve for the given parameters...)

\section{The MeqTree Module}

The MeqTree module is implemented by a software package loosely called {\em
MeqTimba}. MeqTimba consists of the following components:

\begin{itemize}

\item A computational {\em kernel}, mostly implemented in C++ (and using
libraries from AIPS++, FFTW, etc.) The kernel provides all MeqNode classes,
implements basic facilities for creating MeqNodes and connecting them into
trees, and provides a low-level interface for controlling MeqTrees. 

\item A set of {\em I/O agents} to feed the kernel with data and to dispose of
the results. The current set includes agents for reading/writing AIPS++
Measurement Sets (MSs), and also agents for pipelining data over the network.

\item A GUI called the {\em MeqBrowser}, implemented in Python/PyQt.
The browser provides kernel control (bulding trees, attaching MSs, running
trees) and visualization. The browser also provides an interface to the 
stepwise debugging and tree profiling functions of the kernel.

\item A Python-based Tree Definition Language (TDL). TDL scripts can be loaded
and run by the browser, which feeds them to the kernel to construct and run
trees. TDL allows one to define MeqTrees in high-level terms. The typical
turnaround time for modifying and re-executing a TDL-described ME is measured in
seconds. This makes the system remarkably easy to ``play'' with. 

\end{itemize}

\noindent The design of MeqTimba has a number of important highlights:

\noindent{\bf Policy-free:} The kernel is (almost) entirely policy-free; it
operates with very basic concepts (nodes and trees), with mimumum assumptions
about the problem domain. All policy --- and thus the problem domain, complete
with MEs, data formats, etc. --- is defined from the scripting side via TDL.
This makes the system eminently adaptable to new instruments and problems
(including those outside radioastronomy per se). 

\noindent{\bf Data transparency}: each node maintains a {\em state record} that
can be examined from the browser. No MeqTree is ever a black box, and the user
can examine the behaviour of the model to any level of depth or detail.
Node states can also be published into the browser as a tree runs, providing an
execution history. 

\noindent{\bf Visualize everything:} the browser provides built-in tools for
visualizing results and other data structures. Even more importantly, TDL
scripts can define {\em bookmarks} that provide ``canned'' views of the tree
which the user can access with a couple of mouse clicks. In fact, trees can
contain {\em visualization branches} that compute derived quantities that are
not used in calibration per se, but do provide additional insight into
calibration fidelity. 

\noindent{\bf Naturally parallelizable:} MeqTimba has been designed with
parallelization in mind. The current version of the kernel is single-threaded
(although the browser can control it remotely over a network), but future
development will allow for trees to execute in parallel and be distributed
across a cluster. The built-in tree profiler will aid in determining optimal
parallelization strategies.

\section{Current Status And Future Directions}

Over the past year, MeqTimba has been gradually exposed to WSRT data. One
current project (Brentjens 2005) involves custom trees for high-dynamic-range
calibration of a complex field (3C343) which is hard to deal with using
traditional selfcal due to the presence of off-axis bright sources. Performance
on par with existing packages has been demonstrated, and current work aims to
``go where no package has gone before'' and calibrate for finer effects. A
second project aims to provide a set of canned {\em central point source} trees
for online and offline processing of full-polarization WSRT calibrator
observations. On a completely different tack, Willis (2005) has been using
MeqTrees to simulate observations with the projected CLAR telescope,
characterized by a time-variable beam, and show that such a beam may be
successfully calibrated for.

In the near future, we will be applying MeqTrees to data from the new WSRT
Low-Frequency Front Ends (LFFEs), giving a preview of the LOFAR sky. The WHAT
project -- a prototype LOFAR station linked up with the WSRT -- will provide
a very interesting test for MeqTrees, being one of the first examples of a truly
{\em heterogenous} (phased array vs. dish) interferometer.

MeqTrees will play an integral role in LOFAR calibration. The LOFAR {\em Local
Sky Model} (Smirnov \& Noordam 2003; Nijboer et al.\ this volume) will use
MeqTrees to represent sky sources; we see this as the only way to get a handle
on the complex and overcrowded sky seen by LOFAR. The ionosphere is critical at
low frequencies, and will also require a relatively sophisticated model (Noordam
2005). Here again MeqTrees are expected to play a vital role.

\begin{references}
\reference Brentjens, M.\ 2005, \htmladdnormallinkfoot{presentation at SKA WFI
Workshop, Dwingeloo}{http://www.skatelescope.org/pages/news/SKA\_{}WFI2005/scd.pdf}
\reference Hamaker, J.P., Bregman, J.D., Sault, R.J.\ 1996, \aaps, 117, 137
\reference Hamaker, J.P.\ 2000, \aaps, 143, 515
\reference Mevius, M. 2006, \adassxv, \paperref{P.78}
\reference Nijboer, R.J., Noordam, J.E. \& Yatawatta, S.\ 2006, \adassxv,
\paperref{P.57}
\reference Noordam, J.E.\ 2005, 
\htmladdnormallinkfoot{presentation at SKA WFI
Workshop, Dwingeloo}{http://www.skatelescope.org/pages/news/SKA\_{}WFI2005/MIM\%20may\%20202005.ppt}
\reference Smirnov, O.M. \& Noordam, J.E.\ 2003, \adassxiii, 18
\reference Willis, A.G.\ 2005, 
\htmladdnormallinkfoot{presentation at SKA WFI
Workshop, Dwingeloo}{http://www.skatelescope.org/pages/news/SKA\_{}WFI2005/wfi\_{}talk\_{}full.pdf}
\end{references}

%-----------------------------------------------------------------------
%			      References
%-----------------------------------------------------------------------
% List your references below within the reference environment
% (i.e. between the \begin{references} and \end{references} tags).
% Each new reference should begin with a \reference command which sets
% up the proper indentation.  Observe the following order when listing
% bibliographical information for each reference:  author name(s),
% publication year, journal name, volume, and page number for
% articles.  Note that many journal names are available as macros; see
% the User Guide listing "macro-ized" journals.   
%
% EXAMPLE:  \reference Hagiwara, K., \& Zeppenfeld, D.\  1986, 
%                Nucl.Phys., 274, 1
%           \reference H\'enon, M.\  1961, Ann.d'Ap., 24, 369
%           \reference King, I.\ R.\  1966, \aj, 71, 276
%           \reference King, I.\ R.\  1975, in Dynamics of Stellar 
%                Systems, ed.\ A.\ Hayli (Dordrecht: Reidel), 99
%           \reference Tody, D.\  1998, \adassvii, 146
%           \reference Zacharias, N.\ \& Zacharias, M.\ 2003,
%                \adassxii, \paperref{P7.6}
% 
% Note the following tricks used in the example above:
%
%   o  \& is used to format an ampersand symbol (&).
%   o  \'e puts an accent agu over the letter e.  See the User Guide
%      and the sample files for details on formatting special
%      characters.  
%   o  "\ " after a period prevents LaTeX from interpreting the period 
%      as an end of a sentence.
%   o  \aj is a macro that expands to "Astron. J."  See the User Guide
%      for a full list of journal macros
%   o  \adassvii is a macro that expands to the full title, editor,
%      and publishing information for the ADASS VII conference
%      proceedings.  Such macros are defined for ADASS conferences I
%      through XI.
%   o  When referencing a paper in the current volume, use the
%      \adassxii and \paperref macros.  The argument to \paperref is
%      the paper ID code for the paper you are referencing.  See the 
%      note in the "Paper ID Code" section above for details on how to 
%      determine the paper ID code for the paper you reference.  
%

% Do not place any material after the references section

\end{document}  % Leave intact
